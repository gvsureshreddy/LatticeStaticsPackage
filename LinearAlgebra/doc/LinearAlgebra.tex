\documentclass{article}

\usepackage{amsmath}

\newcommand{\Func}[1]{\textbf{#1}}
\newcommand{\var}[1]{\textit{#1}}
\newcommand{\defn}[1]{\textrm{#1}}
\newcommand{\tild}{\ensuremath{\overset{\sim}{~}}}

\newenvironment{LIST}{\begin{list}{$\bullet$}{%
      \setlength{\leftmargin}{2\leftmargin}%
      \setlength{\itemindent}{-1cm}}}%
  {\end{list}}


\begin{document}

\section{Vector class}

This is an arbitrary dimension vector class of type Elm (double).  The
interface is as follows:

\subsection{Data Members}

\begin{LIST}
   \item Elm \var{*Elements\_}\\
    pointer to vector memory.
\end{LIST}

\subsection{Member Functions}

\begin{LIST}
 \item \Func{Vector}();\\
  uninitialized vector of dimension zero
 \item \Func{Vector}(const unsigned\& \var{Cols});\\
  creates a vector of dimension \var{Cols} and does NOT initialize the data
 \item \Func{Vector}(const unsigned\& \var{Cols},const Elm\& \var{InitVal});\\
  creates a vector of dimension \var{Cols} and initializes all entries to
  \var{InitVal}
 \item \Func{Vector}(const Vector\& \var{A});\\
  creates a distinct copy of the existing Vector \var{A}
 \item \Func{Vector}(const Vector3D\& \var{A});\\
  creates a distinct copy of the existing Vector3D \var{A}
 \item \Func{Vector}(const Matrix\& \var{A});\\
  creates a distinct copy of the (row or column) Matrix \var{A}
  
 \item \Func{\tild Vector}();\\
  releases memory and destructs the instance of the Vector
  class.\\[.5cm]

 \item friend Vector\& \Func{operator+}(Vector\& \var{A});\\
  Unitary plus operator
 \item friend Vector \Func{operator+}(const Vector\& \var{A},
  const Vector\& \var{B});\\
  Binary addition operator
 \item friend Vector \Func{operator-}(const Vector\& \var{A},
  const Vector\& \var{B});\\
  Binary subtraction operator
 \item friend Vector \Func{operator-}(const Vector\& \var{A});\\
  Unitary negation operator
 \item friend Elm \Func{operator*}(const Vector\& \var{A},
  const Vector\& \var{B});\\
  Vector dot-product operator
 \item friend Vector \Func{operator\%}(const Vector\& \var{A},
  const Vector\& \var{B});\\
  Cross product operation, only valid for Vectors of dimension 3
 \item friend Vector \Func{operator*}(const Matrix\& \var{A},
  const Vector\& \var{B});\\
  Matrix right multiplication operator
 \item friend Vector \Func{operator*}(const Vector\& \var{A},
  const Matrix\& \var{B});\\
  Matrix left multiplication operator
 \item friend Vector \Func{operator*}(const Elm\& \var{A},
  const Vector\& \var{B});\\
  scalar left multiplication operator
 \item friend Vector \Func{operator*}(const Vector\& \var{A},
  const Elm\& \var{B});\\
  scalar right multiplication operator
 \item friend Vector \Func{operator/}(const Vector\& \var{A},
  const Elm\& \var{B});\\
  scalar division operator \\[.5cm]
  
 \item Elm\& \Func{operator[]}(const unsigned\& \var{i});\\
  Element access operator (with or without bounds checking depending on compile
  time option)
 \item const Elm \Func{operator[]}(const unsigned\& \var{i}) const;\\
  Element access operator (with or without bounds checking depending on compile
  time option)\\[.5cm]

 \item Vector\& \Func{operator=}(const Vector\& \var{B});\\
  Assignment operator
 \item Vector \Func{operator+=}(const Vector\& \var{B});\\
  Incremental addition operator
 \item Vector \Func{operator-=}(const Vector\& \var{B});\\
  Incremental subtraction operator
 \item Vector \Func{operator*=}(const Elm\& \var{B});\\
  Incremental scalar multiplication operator \\[.5cm]

 \item void \Func{Resize}();\\
  change vector dimension to zero
 \item void \Func{Resize}(const unsigned\& \var{Cols});\\
  change vector dimension to \var{Cols}.  No initialization will be performed.
  (Note: if \var{Cols} is equal to the current dimension of the vector then no
  action is taken)
 \item void \Func{Resize}(const unsigned\& \var{Cols},const Elm\& \var{InitVal});\\
  change vector dimension to \var{Cols} and initialize to \var{InitVal}\\[.5cm]

 \item unsigned \Func{Dim}() const;\\
  returns the dimension of the vector
 \item Elm \Func{Norm}();\\
  returns the Euclidean norm of the vector\\[.5cm]
  
 \item friend Vector \Func{SolvePLU}(const Matrix\& \var{A},
  const Vector\& \var{B});\\
  Uses PLU decomposition with Forward and Backwards substitution to solve the
  system of equations \var{A}*x = \var{B}
 \item friend Vector \Func{SolveSVD}(const Matrix\& \var{A},
  const Vector\& \var{B});\\
  Uses SVD decomposition to solve the system of equations \var{A}*x = \var{B}
 \item friend Vector \\Func{SolveSVD}(const Matrix\& \var{A},
  const Vector\& \var{B},const Elm \var{MaxCond});\\
  same as above except \var{MaxCond} sets the largest condition number above
  which the smallest Singular Values are set to zero.  See Numerical Recipes
 \item friend Vector \Func{SolveSVD}(const Matrix\& \var{A},
  const Vector\& \var{B},\\const Elm \var{MaxCond}, const int \var{PrintFlag});\\
  same as above except if \var{PrintFlag}=1 the condition number of \var{A}
  is printed to the terminal \\[.5cm]
  

 \item friend ostream\& \Func{operator$<<$}(ostream\& \var{out},
  const Vector\& \var{A});\\
  output operator
 \item friend istream\& \Func{operator$>>$}(istream\& \var{in},Vector\& \var{A});\\
  input operator, reads \var{Cols} Elms from \var{in} stream
\end{LIST}

\section{Matrix class}

This is an arbitrary dimension matrix class of type Elm (double).  The
interface is as follows:

\subsection{Constants}

\begin{LIST}
 \item define \defn{MAXCONDITION} = 10.0e12\\
  Maximum condition number parameter.  Default value used by the SVD
  decomposition member function
 \item static int \var{MathematicaPrintFlag}\\
  Flag to control stream output.  If =1 then output can be read directly into
  Mathematica. If =0 then output is simple space delimited matrix form
\end{LIST}

\subsection{Data Members}

\begin{LIST}
 \item Elm \var{**Elements\_}\\
  pointer to matrix data which is stored contiguously, just as in normal C
 \item unsigned \var{Rows\_}\\
  Number of rows in the matrix
 \item unsigned \var{Cols\_}\\
  Number of columns in the matrix
\end{LIST}

\subsection{Member Functions}

\begin{LIST}
 \item \Func{Matrix}();\\
  uninitialized matrix of dimension $0 \times 0$
 \item \Func{Matrix}(unsigned \var{Rows}, unsigned \var{Cols});\\
  creates a matrix of dimension \var{Rows} $\times$ \var{Cols} and does NOT
  initialize the data
 \item \Func{Matrix}(unsigned \var{Rows}, unsigned \var{Cols}, Elm
  \var{InitVal});\\
  creates a matrix of dimension \var{Rows} $\times$ \var{Cols} and initializes
  all entries to \var{InitVal}
 \item \Func{Matrix}(const Matrix\& \var{A});\\
  creates a distinct copy of the existing Matrix \var{A}
 \item \Func{\tild Matrix}();\\
  releases memory and destructs the instance of the Matrix class\\[.5cm]

 \item friend Matrix\& \Func{operator+}(Matrix\& \var{A});\\
  Unitary plus operator
 \item friend Matrix \Func{operator+}(const Matrix\& \var{A}, const Matrix\&
  \var{B});\\
  Binary addition operator (safe for overwrite: A=A+B)
 \item friend Matrix \Func{operator-}(const Matrix\& \var{A});\\
  Binary substraction operator (safe for overwrite: A=A-B)
 \item friend Matrix \Func{operator-}(const Matrix\& \var{A});\\
  Unitary negation operator
 \item friend Matrix \Func{operator*}(const Matrix\& \var{A}, const Matrix\&
  \var{B});\\
  Matrix multiplication (save for overwrite: A=A*B)
  
%   ~Matrix();
%
%   // Size Access...
%   unsigned Rows() const {return Rows\_;}
%   unsigned Cols() const {return Cols\_;}
%   
%   // Mathematical Operations...
%
%   friend Matrix operator*(const Elm& A,const Matrix& B);
%   friend Matrix operator*(const Matrix& A,const Elm& B);
%   // Below are defined in corresponding class --------------------
%   friend Vector operator*(const Matrix& A,const Vector& B);
%   friend Vector operator*(const Vector& A,const Matrix& B);
%   friend Vector3D operator*(const Matrix& A,const Vector3D& B);
%   friend Vector3D operator*(const Vector3D& A,const Matrix& B);
%   // -------------------------------------------------------------
%   friend Matrix operator/(const Matrix& A,const Elm& B);
%
%   // Element Access methods
%
%#ifdef CHECK\_BOUNDS
%   // Note: Index checking on Rows but not on Columns....
%   Elm* operator[](unsigned i);
%   Elm* operator[](unsigned i) const;
%#else
%   // Note: NO Index checking
%   Elm* operator[](unsigned i) {return Elements\_[i];}
%   Elm* operator[](unsigned i) const {return Elements\_[i];}
%#endif
%   
%   // Assignment Operations
%
%   Matrix& operator=(const Matrix& B);
%   Matrix operator+=(const Matrix& B) {return *this=*this+B;}
%   Matrix operator-=(const Matrix& B) {return *this=*this-B;}
%   Matrix operator*=(const Matrix& B) {return *this=*this*B;}
%   Matrix operator*=(const Elm& B)    {return *this=*this*B;}
%
%   // Misc. Matrix Operatons
%   
%   Matrix& SetIdentity(unsigned Size=0);
%   Matrix Transpose() const;
%   Matrix Inverse() const;
%   int IsSquare() const {return Rows\_==Cols\_;}
%   int IsNull() const {return (Rows\_==0 || Cols\_==0);}
%
%   // Destructively Resize Matrix
%   // No change if size does not change
%   void Resize(unsigned Rows=0,unsigned Cols=0,Elm InitVal=SENTINAL);
%   
%   // Operations & Etc...
%
%   // Deterimnent
%   Elm Det() const;
%
%   // Set P,L,U to the corresponding matricies of the PLU
%   //   decomposition of A
%   friend void PLU(const Matrix& A,Matrix& P,Matrix& L,Matrix& U);
%
%   // Singular Value Decomposition of A -- Algorithm from Numerical Recipies
%   //
%   // return value - condition number of A
%   // A - mxn matrix to decompose
%   // U - mxn "column-orthogonal" matrix
%   // W - nxn diagonal matrix (singular values)
%   // V - nxn orthogonal matrix
%   //
%   // A = U*W*V.Transpose();
%   //
%   // each W[i][i] < fabs(MAX(W)) / MaxCond; will be set to 0.0
%   // -- this most often reduces error when solving a system of equations
%   // -- that is ill-conditioned (as compaired with a straight SVD or LU
%   // -- decomposition).
%   //
%   // if (PrintFlag); then the condition number of A will be echoed on cerr
%   //
%   // WHENEVER a W[i][i] is explicitly set to 0.0 a message is echoed to cerr
%   // NOTE: this situation may be detected by the calling program by compairing
%   // -- the value of MaxCond with the returned condition number.
%   //
%   friend Elm SVD(const Matrix& A,Matrix& U,Matrix& W,Matrix& V,
%                  const Elm MaxCond=MAXCONDITION,const int PrintFlag=0);
%
%   // SymEigVal -- determine the eigenvalues of A
%   // Diag(eigen values) = B.Transpose()*A*B
%   //
%   // Use Cyclic Jacobi Method -- Ref. "Numerical Analysis" by Pratel pg 440
%   //
%   // Returns vector containing eigenvalues
%   // AND-- Matrix of eigenvectors (as columns) if a pointer is passed
%   // MaxItr - max number of iterations
%   // Tol - tolerance for convergence
%   //
%   // Note: Assumes A is SYMMETRIC
%   friend Matrix SymEigVal(Matrix A,Matrix *B=NULL,const int MaxItr=100,
%                           const double Tol=1.0e-13);
%   
%   // Cholesky Decomposition of Matrix
%   // A=U.Transpose()*D*U
%   //
%   // D - diagonal Matrix
%   //
%   // Assumes Symmetric Matrix (thus uses only Upper Diagonal part of A
%   // Note: will fail if A has EigenValue of 0.0
%   friend void Cholesky(const Matrix& A,Matrix& U,Matrix& D);
%
%   // Return solution x of the linear system A*x=B
%   // Uses PLU decomposition and Forward and Backwards substitution
%   friend Matrix SolvePLU(const Matrix& A,const Matrix& B);
%   
%   // Return solution x of the linear system A*x=B
%   // Uses SVD decomposition
%   //
%   // x = V*W.Inverse()*(U.Transpose()*B);
%   // WHERE: W.Inverse() is actually calculated by hand and any
%   // -- W[i][i] == 0.0 has inverse component 0.0
%   friend Matrix SolveSVD(const Matrix& A,const Matrix& B,
%                          const Elm MaxCond=MAXCONDITION,
%                          const int PrintFlag=0);
%   
%   // Output/Input Functions
%   friend ostream& operator<<(ostream& out,const Matrix& A);
%   friend istream& operator>>(istream& in, Matrix& A);
%
%   static char* Revision();
%};
%
%#endif
\end{LIST}

\end{document}

% LocalWords:  const Cols SolvePLU PLU SolveSVD SVD Func MaxCond int PrintFlag
% LocalWords:  ostream istream MAXCONDITION MathematicaPrintFlag Mathematica
